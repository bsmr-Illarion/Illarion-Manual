
\documentclass[a4paper,11pt]{book}
\usepackage[ngerman]{babel}
\usepackage[utf8x]{inputenc}
\usepackage{fullpage}
\renewcommand{\familydefault}{\sfdefault}
\makeindex
\begin{document}

Ein kostenloses MMORPG mit echtem Rollenspiel

Die Welt ist in Aufruhr. Die Rückkehr der Alten Götter zerrüttete die Reiche, Flüchtlinge aller Völker strömen in die Bastionen der Menschheit im Land Illarion, die verschont geblieben sind von den Entbehrungen der vergangenen Tage. Sechs Edelsteine der Macht waren den Fürsten dieser Bastionen zur Verwahrung gegeben; doch Missgunst, Verrat und Neid sind die alltäglichen Geißeln der Macht.

Illarion ist ein kostenloses Online-Rollenspiel, welches seinen Schwerpunkt auf echtes Rollenspiel legt. Alle Charaktere um dich herum werden sich wie lebendige, atmende Wesen dieser eigenständigen, geheimnisvollen Welt verhalten. Jeder Charakter hat eine eigene Vergangenheit, Ziele, Stärken und Schwächen. Erlebe als edler Ritter ruhmvolle Abenteuer oder führe ein Leben als fleißiger Handwerker, geschäftiger Händler oder charismatischer Priester der Götter.

Illarion vereint ein klassisches Fantasy-Setting mit den Vorzügen einer offenen, persistenten Spielwelt. Deine Entscheidungen und Taten formen und gestalten diese Welt und werden eines Tages die Seiten der Geschichtsbücher füllen. Du wirst dich dem Zauber dieser Welt nicht entziehen können!

Illarion lebt und wird seit über 13 Jahren kontinuierlich weiterentwickelt und den Wünschen der Spieler angepasst.

Es ist jederzeit mit kleineren und größeren Änderungen zu rechnen.

Das Dokument versteht sich daher als Hilfestellung und nicht als allein gültige Programmdokumentation.

Wir spielen miteinander, nicht gegeneinander!

Der Mensch, der die Figur spielt, die im Spiel der Feind der von dir gespielten Figur ist, ist dein Freund! Einfach mal drüber nachdenken!

Du bist nicht der von dir gespielte Charakter.

Neu im Spiel?

Frage Gamemaster oder Spieler, wenn dir was nicht klar ist!

Sprich mit uns im Spiel, im IRC Chat, in TeamSpeak oder über unser Forum.

Umfangreiche Hinweise finden sich auf der Homepage von Illarion.

\tableofcontents

\chapter{Bedienung}

\section{Erste Schritte}

Aller Anfang ist schwer, auch in und mit Illarion.

Bleib einfach einige Tage dabei. Du wirst sehen, auch du wirst dieses Spiel liebgewinnen.

Wenn du das erste Mal Illarion spielst wird es dich auf eine Übungsinsel verschlagen. Die NPC’s erklären dir sehr ausführlich, wie alles funktioniert. Geh einfach weiter nach Osten, wenn du das Prinzip verstanden hast.

In den ersten Stunden deiner Spielzeit bekommen erfahrene Spieler eine Meldung sobald du online gehst. Wer Zeit und Lust hat kommt zu dir. Erschrick nicht, wenn du plötzlich umzingelt bist, die wollen alle nur spielen (helfen).



\begin{tabular}{ l l p{9cm} }
  (1) & Reden & Probiere das Sprechen sowie das Schreien (\#s), Flüstern (\#w) und ooc (\#o). Mit Enter startest und beendest du die Chatfunktion. \\
  (2) & Laufen & linke Maustaste, WASD oder Numpad bewegen dich wohin du willst. Auf der Stelle drehen geht mit Numpad und ALT-Taste. Rennen geht mit Numpad und ALTGR-Taste. \\
  (3) & Inventar & Öffne unten rechts den Button mit der Tasche. Dein Inventar erscheint. Öffne dann die Tasche im Inventar oben links. Verschieb einige Items. \\
  (4) & Kämpfen & Der Charakter, der einen roten Kreis hat wird von dir angegriffen. Mit der rechten Maustaste wählst du ihn an und ab. Versuch das mit einem Schwein. \\
  (5) & Essen & Geh zu einem Apfelbaum und klick ihn mit links an. Im Inventar erscheinen Äpfel. Ist er abgeerntet, Doppelklick auf die gesammelten Früchte.\\
  (6) & Stadt & Entscheide dich für eine Stadt. Es ist egal ob du in Galmair dein Geld dem Don in den Rachen werfen wirst, in Cadomyr für die Ehre der Königin stirbst oder in Runewick vom Erzmagier zum Büchersortieren verdonnerst wirst. Du bist klug genug, dem zu entgehen.
\end{tabular}
   
Jetzt bist du fit für die Welt da draußen.

{\huge Scheu dich nicht zu fragen, wenn dir was unklar ist.}

Unsere wichtigste Hilfefunktion sind unsere Spieler. Und die funktioniert sehr gut. Gib zu erkennen, dass du neu bist. Neben Hilfe bekommst du sicher auch noch das ein oder andere Willkommensgeschenk von dem dein Charakter noch lange träumen müsste, wenn er es selber bezahlen sollte.

\section{Oberfläche}
\begin{tabular}{ l l p{9cm} }
Markierung & Name & Anmerkung \\ \hline 
G  & Chat Bereich & Der Chatbereich kann über das Buch neben der Eingabezeile vergrößert werden.\\

M & Mini Karte & \\

L & Gesundheitszustand & rot – Lebensenergie gelb – Ernährung blau – Mana \\
\end{tabular}
\\

Einzelne Komponenten können über die Tastatur angesteuert werden, sofern nicht gerade der Chat Modus aktiv ist.

\begin{tabular}{ l l}
Taste & Bedeutung \\ \hline
B  &Tasche \\
C  &Skills \\
Q  &Quests \\
I  & Inventar
\end{tabular}
\\

Eigenschaften von Charakteren können durch die Maus und über die ALT GR Taste angezeigt werden. Die Schriftfarbe des Namens zeigt an, um welche Art von Charakter es sich handelt:
\begin{tabular}{ l l p{9cm}}
Schriftfarbe & Charakterart & Anmerkung \\ \hline
Gelb  & Spielercharakter & Es handelt sich hierbei um den Charakter eines anderen Spielers. \\
Blau  & Kooperativer NPC & Kann nicht angegriffen werden, redet aber gerne mit dir. \\
Rot  & Feindlicher NPC & Kann angegriffen werden, wehrt sich aber auch.
\end{tabular}

\section{Inventar}

Das Inventar zeigt an, welche Gegenstrände dein Charakter gerade angelegt hat.

Im oberen Teil sind die Ausrüstungsgegenstände dargestellt, die der Charakter gerade verwendet.

Im unteren Bereich, dem Gürtel kann er einige Gegenstände parat haben. Nur die angelegten Gegenstände haben einen Einfluss auf die aktuellen Fähigkeiten und Fertigkeiten.

Eine Rüstung im Gürtel beeinflusst die Rüstungsstärke nicht.

Zu jedem Item werden der Wert (Einkaufspreis), das Gewicht, die Qualität und die Haltbarkeit dargestellt.

Eventuell siehst du auch noch zusätzliche Informationen, wie den Namen des Handwerkers.

Siehe dazu auch die Kapitel Handel und Gegenstände.

\section{Chat Modus}

Illarion legt Wert auf echtes Rollenspiel. Die eigentliche Interaktion mit anderen Mitspielern und auch NPC erfolgt im Chat Modus.

Über die RETURN Taste wird der CHAT Modus ein- und ausgeschaltet.

\#me  Emote

Du zeigt nicht hörbare Emotes deines Charakters. Der Text wird in Gelb dargestellt.

z.B.  \#me lacht  Î

 \#me stinkt, als wär er in eine Jauchegrube gefallen. \#w  Flüstern

Die gesprochenen Worte sind nur bis zum übernächsten Feld zu hören. Geflüsterter Text wird in Grau dargestellt.

\#s  Schreien

Das gesprochene Wort ist weithin zu vernehmen. Geschriener Text wird in Rot dargestellt.

\#o  OOC (Out of Character)

Du signalisierst deinen Mitspielern, dass hier nicht dein Charakter sondern du selber sprichst. Der Text wird in grau und in Doppelklammern ((Das ist eine ooc Bemerkung.)) dargestellt und ist wie geflüsterter Text nur bis zum übernächsten Feld zu hören.

(( … )) Du signalisierst deinen Mitspielern, dass hier nicht dein Charakter sondern du selber sprichst. Bitte verwende ooc nur wenn notwendig. [1] (z.B. ((Ich muss in 5 min Schluss machen.)) ) Ansonsten hat Illarion auch einen IRC Chat Kanal. Sonstiges

\#i  Vorstellen

!gm  Nachricht an GM

Die Komponenten der Oberfläche können direkt aufgerufen oder ausgeblendet werden.

Ctrl B  Tasche

Ctrl Q  Quests

Ctrl I  Inventar

\section{Sprachen}

Es gibt rassenspezifische Sprachen. Diese wird nur von Angehörigen der jeweiligen Rasse gesprochen und verstanden. Im Gespräch wird eine andere Sprache durch eckige Klammern gekennzeichnet z.B. [elf].

!lcommon allgemeine Handelssprache

!lelf

Elfensprache

!lhuman

Sprache der Menschen

!ldwarf

Sprache der Zwerge

!lliz

Echsensprache

!lhalb

Sprache der Halblinge

Für Deutsch und Englisch haben sich die Bezeichnungen alte Sprache (Deutsch) und neue Sprache (Englisch) eingebürgert. [2]

\section{Bewegen}

\subsection{Laufen}

Du kannst deinen Charakter mit der Maus oder der Tastatur über die Karte bewegen.

Bei gedrückter linker Maustaste folgt der Charakter dem Mauszeiger. In Grenzen weicht er dabei Hindernissen aus.

Mit Doppelklick auf ein freies Feld (Gras, Weg, Fels, Pflaster) wandert der Charakter dorthin.

.

Tasten WASD  Haupthimmelsrichtungen

Pfeiltasten  Haupthimmelsrichtungen

Zwei Tasten können gleichzeitig genutzt werden

Numboard  Haupt- und Nebenhimmelsrichtungen

ALT und Taste  Charakter dreht in die Richtung ohne Laufen

ALT GR und Taste  Charakter rennt

Wenn der Charakter überladen ist (Meldung \"Eine Last bremst dich!\") kann nur noch gewandert werden.

Einige der Tastaturkombinationen können anderweitig vom System belegt sein! (z.B. ALT GR und  ×

\subsection{Teleporter}

Für lange Strecken finden sich vor den Stadttoren und am Gasthaus zur Hanfschlinge Teleporter. Für 10 Silber bringen sie dich zu einem der anderen Teleporter.

Teleporterstationegasthof zur Hanfschlinge

\subsection{Portalbücher}

Für 20 Silber verkaufen einige Händler auch Portalbücher. Mit diesen kann man an beliebiger Stelle ein temporäres Portal zu einem der Teleporter errichten, das beliebig viele Personen benutzen können.

Das Portal entsteht irgendwo in der Nähe dessen, der es aufruft. Leider sind die Magier, die diese Bücher erschaffen, nicht immer sorgfältig, so dass ein Portal auch schon mal einige Schritte entfernt hinter einer Mauer oder über dem Wasser erscheint. Am besten ist, du suchst dir einen freien Platz.

Feindliche NPC scheuen Magie, sogar Magier. Du brauchst keine Angst zu haben, dass der Skelettdrache, der hinter deinem Charakter her ist, plötzlich mitten in einer Stadt steht.

\section{Gegenstände bewegen und lagern}

Um Stapel von Gegenständen zu teilen, die Shift Taste beim Verschieben drücken.

Gegenstände können auch über den entsprechenden Button im Inventar oder über die Kurztaste P aufgehoben werden.

ACHTUNG

Solltest du einen Gegenstand oder einen Stapel von Gegenständen nicht aufnehmen können, ist er entweder fest mit dem Boden verbunden oder zu  schwer.

Vorsicht bei der Übergabe großer Warenmengen, insbesondere in Taschen. Es gibt niemanden, der 200 Erzstücke heben kann! Eine volle Tasche ist schnell aus der Kiste raus. Es dauert, sie in kleinen Stücken zurückzuholen.

Jede Siedlung besitzt ein eigenes Depotsystem. Hier kannst du alle Items unterbringen, die du nicht mit dir herumtragen willst.

Im Depot gibt es keine Gewichtsbeschränkungen. Taschen helfen dir, Ordnung im Depot zu halten. Volle Taschen in Taschen sind nicht möglich. Außer bei Münzen sind Stapel auf 1000 Items beschränkt.

\section{Kämpfen}

Das Leben auf Illarion ist nicht immer friedlich. So manche Auseinandersetzung lässt sich nur mit Gewalt lösen. Als besonders resistent gegen vernünftige Argumente haben sich die verschiedenen NPC’s erwiesen, die die Wildnis dicht bevölkern.

Um einen Charakter anzugreifen ist dieser mit der rechten Maustaste zu markieren.

Nicht jede Waffe ist gegen jede Rüstung gleichermaßen effektiv.

Natürlich haben persönliche Erfahrung und Qualität von Waffen und Rüstungen einen Einfluss. Erfahrene Krieger können deinem Charakter sicher weiterhelfen.

Die folgende Tabelle gibt einen groben Hinweis auf die Waffenwirkung

 Leichte Rüstung  Mittlere Rüstung  Schwere Rüstung Ringen  guter Schutz  normaler Schutz  schlechter Schutz

Stichwaffen  normaler Schutz  schlechter Schutz  guter Schutz

Hiebwaffen  schlechter Schutz  guter Schutz  normaler Schutz

Schlagwaffen,  guter Schutz  normaler Schutz  schlechter Schutz

Schusswaffen  normaler Schutz  schlechter Schutz  guter Schutz

NPC sind nicht besonders helle. Sobald du den Bereich ihrer Waffenwirkung verlässt vergessen sie dich und brauchen einige Zeit, bis sie dich wieder finden und angreifen. Solange du in Bewegung bist brauchst du außer ein paar seltenen Bogenschützen niemanden zu fürchten, der nicht direkt neben deinem Charakter steht.

Je stärker der Charakter verwundet ist, desto langsamer wird er sich bewegen können.

Kurz bevor ein Charakter stirbt stolpert er zurück. Damit müssen Kämpfe zwischen

Charakteren nicht tödlich enden. NPC’s finden dich meist bevor sich dein Charakter von dem Schock erholt hat. Dann gibt es keine Rettung mehr.

War ein Gegner zu stark stirbt dein Charakter. Zum Glück ist Cherga (die Göttin der Toten) gnädig oder faul und schickt dich zurück. Es wird eine Weile dauern, bis du von deinen schweren Wunden geheilt bist. Es ist nichts bekannt, was die Heilung beschleunigen kann, wenn du direkt von Cherga kommst.



NPC’s können auch rollenspieltechnisch sofort angegriffen werden.

 Kämpfe zwischen Charakteren sollten immer auf gegenseitigem Einvernehmen der beteiligten Spieler beruhen. Das heißt nicht, dass der gespielte Charakter der gleichen Meinung sein muss.

Endgültig sterben kann ein Charakter nur, wenn der Spieler das rollenspieltechnisch umsetzt. Es ist nicht zulässig einen Charakter ohne Einverständnis des Spielers irreversible zu schädigen. (z.B. \"\#me schlägt ihm den Kopf ab.\")

\section{Handwerk}

Durch Kämpfen kann man viel Geld erlangen. Wirklich reich wird man als Handwerker.

Willst du Rohstoffe gewinnen benutze das entsprechende Werkzeug da, wo die Rohstoffe vorkommen. Nutzt du zum Beispiel eine Angel am Wasser kann es sein, dass du Fische fängst.

Es soll vorkommen, dass du nicht nur das findest, was du gesucht hast. Manch weggeworfenes oder verlorenes Item kommt zum Vorschein. Darunter befinden sich auch solch seltene Dinge wir Schatzkarten, magische Edelsteine oder reine Elemente. Allerdings kann es auch passieren, dass du einen unerwünschten Bewohner erweckst, der deinen Eingriff in seinen Lebensraum gar nicht gut findet.

Um Rohstoffe weiter zu verarbeiten benötigt der Charakter das richtige Werkzeug in der Hand, einen Platz zum Arbeiten und die erforderlichen Rohmaterialien.

Bediene das Werkzeug und du erfährst, was dein Charakter alles herstellen kann, wie lange und was er dazu braucht.

Die Werkbänke, Öfen und sonstigen Werkzeuge stehen den Einwohner einer Stadt kostenlos zur Verfügung. Wills du in einer anderen Stadt arbeiten musst du eine entsprechende Lizenz erwerben.

Primäre Handwerke sind gebräuchlich in der jeweiligen Stadt und werden im vollen Maß unterstützt. Jedes Werkzeug kann von den Händlern besorgt werden und alle statischen Werkzeuge sind vorhanden.

Alle primären Ressourcen für das Handwerk sind auf dem Gebiet vorhanden. Jedes Handwerk gibt es als ein primäres Handwerk für eine Stadt. NPCs kaufen und verkaufen alle Sachen (auch Ressourcen), die von jenem Handwerk hergestellt werden können.

Sekundäre Handwerke sind nicht gebräuchlich in der jeweiligen Stadt, aber möglich. Jedes Werkzeug kann von den Händlern besorgt werden und alle statischen Werkzeuge sind vorhanden.

Eine beschränkte Anzahl von Ressourcen für das Handwerk sind auf dem Gebiet vorhanden.

\section{Handel}

Wenn du mit einem Mitspieler handelst, so lege einfach die vereinbarten Waren oder Geldstücke auf den Boden neben den Charakter und hebe die dir überreichten Waren auf.

oder  [3]

Waren, die verkauft werden können, werden durch eine Münze gekennzeichnet. Doppelklicke auf den Item und er wird verkauft.

Es wird immer der ganze angeklickte Stapel verkauft. Willst du nicht den ganzen Stapel verkaufen, teile ihn vorher

[4]

Eine silberne Münze zeigt, dass es irgendwo noch jemanden gibt, der mehr bezahlt. Eine goldene Münze zeigt, dass du bei keinem NPC mehr für diesen Artikel bekommst.z.B. magische Edelsteine, reine Elemente, seltene Kräuter

Du wirst keinen NPC Händler finden, der einen höheren oder niedrigeren Preis für ein Item verlangt. Der Preis entspricht dem im Tooltip angezeigten Wert. 

NPC Händler zahlen je nach Stadt 10 oder 5 \% des Wertes eines Items, wenn du es verkaufen willst. 

Die Waren und Preise der Händler orientieren sich an den Gewerken, die in dieser Stadt vertreten sind, Von NPC Händlern verkaufte Waren sind immer von durchschnittlicher Qualität.

Der Handel mit Mitspielern verspricht in jedem Fall mehr Spaß durch Rollenspiel. Außerdem tut er dem Geldbeutel deines Charakters gut.

\section{NPC und Quests}

NPC’s können grob in kooperative und feindliche aufgeteilt werde.

Feindliche sind am roten Text zu erkennen und greifen deinen Charakter an sobald sie seiner gewahr werden.

Kooperative NPC können nicht angegriffen werden.

Fast alle kooperative NPC haben auch Aufgaben zu vergeben. Hier fehlt etwas Holz fürs Feuer, dort muss ein Kuchen gebacken werden oder eine Nachricht überbracht werden.

Sprich einfach mit ihnen, sie reagieren auf Schlüsselwörter.

Hilfe  Du erhältst eine kurze Information über den NPC und eine Liste wichtiger Schlüsselwörter.

Quest  Der NPC hat eventuell eine Aufgabe, die er deinem Charakter übertragen will.

Handel  Der NPC wird dir seine Auswahl an Waren präsentieren.

Erzähl was Der NPC wird dir eine Geschichte erzählen. Ob du sie glaubst, bleibt dir überlassen.

Die jeweiligen Stadtoberen scheinen Wert darauf zu legen, dass du ihren festen Bewohnern hilfst. Für viele Aufgaben von NPC’s aus deiner Stadt

bekommst du auch Rangpunkte in deiner Stadt.

\section{Sonstiges}

\subsection{Geld}
100 Silber = 1 Gold

In jedem Ort gibt es einen NPC, der viele kleine Münzen in wenige große ohne Aufschlag tauscht.

Geldstücke sind die einzigen Items, die nichts wiegen und in beliebiger Menge transportiert werden können.

\subsection{Kulturtechniken}

Es sollte klar sein, dass ein Apfel den Magen nicht so nachhaltig füllt, wie ein

Kirschkuchen. Speist du besser wird sich das eine Zeitlang auf einige deiner Fähigkeiten positiv auswirken. Je besser das Essen desto länger fällt die Wirkung aus.

Benutze die Speise und dein Charakter wird zufrieden mit dir sein.

Nachrichten schicken

Tauben sind in Illarion unheimlich. Sie finden jeden, egal, wo er sich gerade befindet.

Sende einfach eine PM an den Forenaccount des Spielers.

Einige Charakter verwenden auch Ratten, Möwen oder sonstige Techniken. Das Prinzip ist das Gleiche.

Benutze einen vollen Eimer Wasser. Dein Charakter wird sich gleich viel sauberer finden.

Wunden verheilen deutlich schneller als im realen Leben. Allerdings sollte dein Charakter nicht gerade am Verhungern sein.

Trotzdem haben Alchemisten Tränke entwickelt, die den normalen Heilungsprozess erheblich beschleunigen.

Wie im richtigen Leben kennst du die Namen deiner Gegenüber nicht automatisch. Auch dein Gegenüber kennt den Namen deines Charakters nicht. Du kannst dich vorstellen. Über den Chatbefehl \#i wird dein Name für alle, die sich in deiner unmittelbaren

[5] Nähe befinden sichtbar.

Wenn dein Charakter eine einzelne Münze in die Hand nimmt, kann er diese werfen und die Entscheidung dem Gott des Chaos und Glückspiels Nargun überlassen. Es gibt auch Würfelbecher zum Vorzugspreis von 9999 Kupfer zu kaufen. Damit können mehrere W6 oder W20 Würfel geworfen werden.

Das Spielen eines Instrumentes ist eine Kunst, allerdings eine ziemlich brotlose. Um damit Geld zu verdienen solltest du lange üben und dann andere Charakter davon überzeugen, dir Geld zu geben.

\subsection{Beschriften}

Taschen, Alchemierezepte und Flaschen können beschriftet werden.

Nimm dazu eine Feder in eine Hand und das zu beschriftende Item in die andere (Tasche über die Schulter). Benutze die Feder.



\section{Kurztasten}

Die folgende Seite beinhaltet eine Zusammenfassung der Tastaturbefehle.

B  Tasche C  Skills

Q  Queste

I  Inventar

ALT GR  Zeige Name und Gesundheit

\#me  Emote

\#w  Flüstern

\#s  Schreien

\#o  OOC

\#i  vorstellen

!gm  GM kontaktieren

Tasten WASD

Haupthimmelsrichtungen

Pfeiltasten

Haupthimmelsrichtungen

Numboard

Haupt- und Nebenhimmelsrichtungen

ALT und Taste

Charakter dreht in die Richtung ohne Laufen

ALT GR und Taste

Sonstiges

Charakter rennt

P Maus

Plündern, hebe alles auf, was sich direkt neben vor und hinter deinem Charakter befindet.

Links halten

Mauszeiger folgen

Links Klick

Eigenschaft

Links Doppelklick

Item nutzen

zu dem leeren Feld laufen

Rechts Klick

angreifen

Shift Maus

Entstapeln

\chapter{Rollenspiel}

Es gibt eine Vielzahl von Ansätzen und Meinungen, die Rollenspiel beschreiben. Diese ist

Hintergundgeschichte  Du musst keinen Roman schreiben, aber einige grundlegende Eckpunkte solltest du haben. Die Löcher in der Geschichte werden sich im Laufe der Zeit auch füllen.

Vor- und Nachteile

Niemand ist nur ein strahlender Held und keiner verliert immer. Dein Charakter sollte positive als auch negative Eigenschaften haben.

Ziele

Gib ihm einen Traum, irgendetwas was er unbedingt möchte.

Phobie

Gib ihm etwas, was er hasst oder fürchtet.

Unverwechselbarkeit

Dein Charakter sollte irgendeine persönliche Eigenart haben,

zum Beispiel eine leise Stimme, eine besondere Grußformel oder eine Vorliebe für gelbe Hüte

Berücksichtige, dass er in Illarion ungelernt anfangen wird. Ein erfahrener Krieger oder Handwerker kann er am Anfang nicht sein. [6]

Gib deinen Mitspielern die Chance zu reagieren. Ziehe ihnen nicht am Ärmel sondern versuch es nur. Ein guter Rollenspieler wird darauf eingehen.

Spiele so, dass die anderen Spieler auch Spaß am Spiel haben können.

Zwinge niemanden etwas auf. Insbesondere Diebstahl und Angriffe sind schwer zu spielen und bedürfen zwei Spieler, die daran Spaß finden.

Ein Bösewicht spielt sich immer schwerer als ein normaler Charakter. Du musst vor allem bereit sein, ihn letztendlich seiner gerechten Strafe zuführen zu lassen.

Lass dich nicht täuschen. Hinter einem bärbeißigen Zwerg, gierigem Händler, blutrünstigem Ork oder gleichgültigem Magier steht ein Mensch, mit dem du Pferde stehlen würdest.

Neu im Spiel?

Frage Gamemaster oder Spieler, wenn dir was nicht klar ist!

Sprich mit uns im Spiel, im IRC Chat oder über unser Forum.

Du glaubst, du hast keine Ahnung vom Rollenspiel?



Eine umfangreiche Rollenspielanleitung mit Hintergrundgeschichte und vielen Hinweisen findest du auf unserer Webseite.



\chapter{Skills und andere Zahlen}

\section{Skills}

Durch Arbeit und Kampf steigen deine Fähigkeiten in den entsprechenden Skills.

Ein Levelaufstieg wird durch eine magische blaue Spirale angezeigt.

Der aktuelle Stand wird im Skillfenster angezeigt. Mehr als 100 Punkte in einer Fertigkeit kannst du nicht erreichen.

Die in einem Zeitraum erlangbaren Skillpunkte sind beschränkt.

Dein Charakter lernt genauso viel, wenn du 2 Stunden Monster kloppst als  wenn du 1,5h rollenspielst und eine halbe Stunde Monster jagst.

Ein Charakter lernt nur dann wirklich gut, wenn man ihn mit Items ausgestattet hat, die seinem Können entsprechen.

[7]

Man lernt nichts, wenn die verwendete Waffe oder Rüstung einen höheren Level als der eigene Skill hat.

Man lernt nichts mehr, wenn das bearbeitete Item oder der bekämpfte Gegner über 20 Punkte schwächer als der eigene Skill ist.

\section{Gegenstände}

In jeder Stadt gibt es einen NPC der dir gegen entsprechende Gebühr deine Ausrüstung repariert. Dabei wird nur die Haltbarkeit auf nagelneu heraufgesetzt. Die Qualität bleibt unverändert. Frage ihn einfach mal.

perfect

excellent

very good

good

normal

average

bad

very bad

awful

horrible

nagelneu, funkelnd

brand new, sparkling

neu, strahlend

new, shiny

fast neu, glänzend

almost new, glittery

gebraucht, gebraucht

used, used

leicht abgenutzt, angekratzt

slightly scraped, slightly scratched, slightly frayed

abgenutzt, zerkratzt

scraped, scratched, frayed

sehr abgenutzt, matt

highly scraped, highly scratched, highly frayed

old

rostig, morsch, fadenscheinig, stumpf

rusty, rotten, threadbare, tarnished

klapprig, zerfallend, zerfetzt, brüchig

\section{Ränge}

Jeder Charakter kann durch Aktivitäten zugunsten seiner Stadt durch NPC’s und Game Master Rangpunkte erhalten. Die Rangpunkte dienen dazu in den Rangstufen aufzusteigen.

Der Aufstieg in den unteren Rangstufen erfolgt automatisch. Danach muss ein GM den Aufstieg bestätigen.

1  Serf, Serf  Hörige, Höriger

2  Servant, Recruit  Dienerin, Rekrut

3  Maid, Page  Magd, Page

4  Abigail, Squire  Zofe, Knappe

5  Dame, Knight  Hofdame, Ritter

6  Lady, Lord  Freifrau, Freiherr

7  Baroness, Baron  Baronin, Baron

8  Countess, Count  Gräfin, Graf

9  Earl, Earl  Fürstin, Fürst

10  Duchess, Duke  Herzogin, Herzog

11  Queen, King  Königin, König

1  Tramp, Tramp  Rumtreiberin, Rumtreiber

2  Assistant, Assistant  Gehilfin, Gehilfe

3  Pedlar, Pedlar  Hausiererin, Hausierer

4  Grocer, Grocer  Krämerin, Krämer

5  Merchant, Merchant  Kauffrau, Kaufmann

6  Financier, Financier  Finanzier, Finanzier

7  Patrician, Patrician  Patrizierin, Patrizier

8  Mogul, Mogul  Mogulin, Mogul

9  Magnate, Magnate  Magnatin, Magnat

10  Tycoon, Tycoon  Tycoon, Tycoon

11  Don, Don  Don, Don

1  Novice, Novice

Novizin, Novize

2  Apprentice, Apprentice

Anwärterin, Anwärter

3  Student, Student

Studentin, Student

4  Scholar, Scholar

Gelehrte, Gelehrter

5  Master, Master

Magister, Magister

6  Doctor, Doctor

Doktorin, Doktor

7  Docent, Docent

Dozentin, Dozent

8  Professor, Professor

Professorin, Professor

9  Dean, Dean

Dekanin, Dekan

10 Rector, Rector

Rektorin, Rektor

11 Archmage, Archmage

Erzmagierin, Erzmagier

Es halten sich hartnäckig Gerüchte, dass mancher seine Dokumente hat fälschen lassen und sich einen Titel gekauft hat.

Die Stadtoberen sehen es nicht gern, wenn jemand ihre Stadt verlässt. Wer umzieht und Bürger einer anderen Stadt wird verliert alle seine Rangpunkte und fängt wieder bei 0 an.

\section{Magische Steine}

Aus dem Steuereinkommen der Städte erhalten die Bürger monatlich magische Steine, wenn sie einen Rang innehaben. Je höher der Rang, desto mehr magische Steine erhält der Charakter.

Die Steine können dazu verwendet werden, Waffen wirksamer zu machen.

Es gibt insgesamt 6 verschiedene magische Steine.

[8]

Amethyst

lila

(Cadomyr)

Obsidan

schwarz

(Galmair)

Rubin

rot

(Runewick)

Saphir

blau

(Galmair)

Smaragd

grün

(Runewick)

Topas

gelb

(Cadomyr)

Level 1

latent  latent

Level 2

bedingt  limited

Level 3

leicht  slight

Level 4

mäßig  moderate

Level 5

durchschnittlich average

Level 6

bemerkenswert notable

Level 7

stark  strong

Level 8

sehr stark  very strong

Level 9

unglaublich unbelievable

Level 10

einzigartig  unique

Steine können durch einen NPC zu stärkeren Steinen kombiniert werden. Es sind jeweils 3 Steine der nächstniedrigeren Levels erforderlich um einen höherwertigen Stein zu erschaffen. Höherwertige Steine können nicht wieder geteilt werden.

Steine können in Waffen eingesetzt werden und verbessern die Angriffswirkung der Waffe um jeweils einen Punkt pro Stein und Level. Befinden sich alle 6 verschiedene Steine in der Waffe, verstärkt sich die Wirkung.

Verstärkung [%] = Summe der Level aller Steine + Level des schwächsten Steins *12

\section{Abenteuergilde}

Eine Gruppe Abenteurer hat an interessanten Stellen Markierungsteine platziert. Man sagt, dass es über 350 sein sollen.

Sie finden sich neben dem Stuhl der Königin genauso wie tief im finstersten Dungeon.

Du kannst sie sammeln (anklicken). Eine Bestenliste zeigt, wie weit dein Charakter die Welt Illarions bereits bereist hat.

Wenn du einige Steine zusammen hast wird dir die Gilde Belohnungen anbieten. Empfohlen ist bei 5 Steinen die Feder, mit der du Taschen beschriften kannst.

\section{Arena}

In jeder Stadt haben die Bewohner die Chance ihre Kräfte gezielt mit Monstern beliebiger Stärke zu messen. In der Arena finden sie einen geeigneten Platz dafür. Unter den Augen von Zuschauern können sie hier ihre wahren Fähigkeiten im Umgang mit Schwert und Rüstung unter Beweis stellen.

Gegen eine Gebühr entlässt der Arenameister das entsprechende Biest auf die Kampffläche. Kann es nicht besiegt werden so wird es nach einiger Zeit wieder eingefangen.

Für die bei den Arenakämpfen errungenen Siege existiert eine eigene Bestenliste.

\section{Schatzkarten}

Hin und wieder findet man Karten, auf denen eine Stelle irgendwo in Illarion markiert ist.

Wenn du dort mit einer Schaufel oder

Spitzhacke gräbst, findest du vergrabene Schätze. Doch sei vorsichtig, alle Schätze sind bewacht und je größer der Schatz, desto stärker sind auch die Wachen.

Am besten du lädst Freunde und Bekannte ein, dich auf deiner Suche zu begleiten.

Beispiel1

2.6.2.3.2.0  Î (2+6+2+3+2) = +15% Wirkung

Beispiel 2

2.5.2.3.2.2  Î (2+5+2+3+2+2)+12*2 = +34% Wirkung

\section{Zeitrechnung und Kalender}

Die Zeit vergeht in Illarion 3-mal schneller als im reellen Leben. 3 Tage in Illarion entsprechen einem normalen Tag

[9]

Elos

Monat der Magie

Tanos

Monat der Fluten

Zhas

Monat der Treue

Ushos

Sommer

Monat der Aussaat

Siros

Monat der Liebe

Ronas

Monat der Freigiebigkeit

Bras

Monat der Opferung

Eldas

Herbst

Monat des Fastens

Irmas

Monat des Handwerks

Malas

Monat des Jägers

Findos

Monat der bildenden Künste

Olos

Winter

Monat der Ernte

Adras

Monat der Trunkenheit

Naras

Monat der vier Winde

Chos

Monat des Gedenkens

Mas

Monat des Blutes

\chapter{Die Welt Illarions}

\section{Die Welt}

Illarion ist ein vielleicht bedeutender, vielleicht auch unbedeutender Winkel der Welt. Die Welt selber ist größer und beherbergt so manche Geschichte, die zu erzählen wäre.

Illarion selber bildet eine Halbinsel, die nach Westen von einem unüberwindlichen Gebirge und nach Norden von einer ebenso unüberwindlichen Mauer begrenzt wird. Wer und warum die Mauer errichtet hat liegt im Dunkel der Geschichte verborgen.

17 Bisher konnten wir uns noch nicht einigen, wo genau Illarion liegt.

\section{Siedlungen}

\subsection{Übersicht}

Illarion hat 4 Siedlungszentren mit eigener Infrastruktur und speziellen Vor- und Nachteilen.

Jedes Siedlungszentrum besitzt:

Teleporter

Eigenes Depot

Markt mit NPC Händlern

Respawnpunkt für eigene Bürger (Kreuz)

Jede Stadt besitzt außerdem:

Werkstätten für die lokalen Gewerke

Arena

Alchemiewerkstatt



Das Kreuz

\subsection{Taverne zur Hanfschlinge}

Rückzugsort für Freie und Vogelfreie Wer auf Steuern verzichten will und auf den Schutz einer Siedlung pfeift findet hier sein Auskommen.

Es gibt keine Regierung aber Borgate schenkt gegen Geld gerne Bier aus. Vorteile

Keine Steuern

/Kein Handwerk

/Keine magischen Steine


\subsection{Cadomyr}

Stadt der Ehre und Ritterschaft

Grafen, Baronessen und Ritter sollten sich hier am Wohlsten fühlen.

Regiert von Ihrer Majestät Königin

-Fischer

-Sandgräber

-Goldschmiede

-Edelsteinschleifer

-Glasbläser

/Keine Nahrungsproduktion außer Fisch

/Keine Holzverarbeitung

Die Farben von Cadomyr sind rot / weiß.

Das Wappen zeigt einen gehörnten Löwen.

Menschen und Echsen haben diese Stadt gegründet.

\subsection{Galmair}

Stadt des Geldes

Schmiede, Händler

und Bergarbeiter bevorzugen diese Stadt

Regiert von Don Vallerio Guillamo

-Schmiede

-Bergbau

-Steinmetz

-

Fehlende Handwerke und

/Keine Schneiderei

/Kein Sand

/Keine Nutztiere

/Kein Honig

Galmair wurde von Zwergen und Orks besiedelt.

\subsection{Runewick}

Seestadt im Südosten

Gelehrte und ihre

Schüler fühlen sich in den weitläufigen Bibliotheken gut aufgehoben.

Regiert von Erzmagier Elvaine Morgan

-Landwirtschaft und

Nahrungsproduktion

-Holzverarbeitung

-Schneiderei

-

Fehlende Handwerke und Ressourcen

/Keine Schmiede

/Keine Goldschmiede

/

Die Farben von Runewick sind Blau / Silber.

Gebaut wurde die Stadt ursprünglich von Halblingen und Elfen.

\section{Wildnis}

Die zur Verfügung stehende Gegend ist sehr groß und so mancher verborgene Winkel harrt seiner Entdeckung. Doch sei auf der Hut, auch Monster lieben dunkle Wälder, weite Wüsten und verlassene Gehöfte.

\section{Geschichte}

Die Geschichte Illarions reicht viele tausend Jahre zurück. Auf der Homepage findest du die bekannten Unterlagen.

Die neuere Geschichte findest du in den Zeitungen und Bibliotheken. Und vielleicht wirst du einmal Teil dieser Geschichten für kommende Generationen werden.

\section{Götter}

\subsection{Allgemein}

Die Bewohner Illarions sind sehr religiös. Kaum eine Handlung, bei denen die Götter nicht angerufen oder gar ihre Hand im Spiel haben. Die Götterwelt, bestehend aus 5 Alten und 11 Jungen Göttern, ist faszinierend und umfangreich, so dass jeder seinem persönlichen Gott huldigen kann.

Detaillierte Informationen und Geschichten finden sich auf der Homepage und in den Büchern der Bibliotheken.

\subsection{Die Alten}

Die alten Götter haben Illarion geschaffen. Jedes Volk hat abweichende Ansichten darüber wie dies geschah.

Die Ältesten wählten Ihr Anhänger selbst aus, mit einer Ausnahme: Tanora (Zelphia).

Ushara

Göttin der Erde

Brágon

Gott des Feuers

Eldan

Gott des Geistes

Tanora

Göttin des Wassers

Findari

Göttin der Luft

\subsection{Die Jungen}

Erschaffen von den Alten, um die Völker von Illarion zu führen und zu beschützen, sind sie sehr in die Geschehnisse involviert und ergreifen Partei im fortwährenden Kampf der Kräfte auf Illarion. Anstatt die Völker zu führen und ihr friedliches Zusammenleben zu gewährleisten, nehmen sie aktiv Anteil am Kampf der Völker, um Einfluss und Macht zu erlangen.

Nargùn

Gott des Chaos

Elara

Göttin des Wissens und der Weisheit

Adron

Gott des Weines und der Feste

Oldra

Göttin der Fruchtbarkeit und des Lebens

Cherga

Göttin der Geister und der Unterwelt

Malachín

Gott der Jagd und der Schlachten

Irmorom

Gott des Handels und des Handwerks

Sirani

Göttin der Liebe und der Freude

Zhambra

Gott der Freundschaft und des Vertrauens

Ronagan

Gott der Diebe und der Schatten

Moshran

Gott des Blutes und der Gebeine

\subsection{Weihen}

Götter sind eigen und wollen gehuldigt werden. Dafür gibt es überall Altäre. Jeder Gott hat mindestens einen Altar.

Ein Charakter kann sich dort einem Gott weihen.

Dieser ist sein bevorzugter Gott. Das heißt nicht, dass er alle anderen Götter verneinen muss.

Um die Weihe durchzuführen, opfere die am Altar des jeweiligen Gottes angegebenen Gegenstände.

\section{Magie}

[10]

\subsection{Allgemein}

Die Welt Illarions ist magisch. Nichts Vergleichbares in der realen Welt kommt den Phänomenen nahe, die für die Bewohner dieser Welt alltäglich sind.

Mana ist ein Zustand der Umgebung. Alles ist von ihr mehr oder weniger durchdrungen.

Normalerweise ist Mana chaotisch, ändert sich ständig und bleibt damit wirkungslos. Wenn Mana strukturiert wird kann sie Wirkungen beliebiger Art entfalten.

Jeder besitzt die Fähigkeit Mana zu bemerken, zu beeinflussen und zu strukturieren. Diese Essenz ist bei Magiern besonders stark ausgeprägt und wird im Laufe ihrer Ausbildung trainiert.

Mana kann durch verschiedene Methoden, genannt Magie, beeinflusst werden:

21

\subsection{Runenmagie}

Eine eher einfache Form der Manabeeinflussung. Vorgeprägte reflexartige magische Handlungen ausgelöst durch das Aufsagen von Runen. Die erzeugten Strukturen sind niemals dauerhaft sondern lösen sich nach kurzer Zeit wieder auf. Runenmagie ist derzeit nicht möglich.

\subsection{Artefaktmagie}

Die magischen Handlungen sind in Gegenstände eingeprägt und werden durch einfache Handlungen ausgelöst. Die Nutzung von Artefakten bedingt das Vorhandensein von Essenz beim Handelnden. Eine magische Ausbildung ist nicht erforderlich.

Artefakte können durch spontane oder zielgerichtete Magie entstehen.

Bekannte magische Artefakt sind Portalbücher und magisch verstärkte Waffen.

\subsection{Ritualmagie}

Die Manabeeinflussung erfolgt zielgerichtet durch diverse zusammenhängende Handlungen. Da bei Ritualen die magischen Gestaltungskräfte mehrerer Personen als auch gerade bereitstehender anderer Wesen gebündelt werden können sind mächtige und zum Teil sehr dauerhafte Wirkungen möglich. Im Laufe der Jahrhunderte haben Magier sehr viele verschiedene Rituale entwickelt.

Sichtbare Ergebnisse magischer Rituale sind zum Beispiel die dauerhaften Stadtportale.

Die Durchführung von Ritualen muss immer mit GM’s abgesprochen werden.

Es gehört zum guten Stil, dass Rituale erst durchgeführt werden, wenn in den Bibliotheken Schriften existieren, die das Ritual auch beschreiben.

\subsection{Spontanmagie}

Unerklärliche Ereignisse und Phänomene können zum Teil auf nicht zielgerichtetes Wirken von Magie zurückgeführt werden. Entweder haben Wesen unbeabsichtigt alle für ein Ritual erforderlichen Handlungen ausgeführt oder es ist zu einer zufälligen Ordnung in der sich ständig bewegenden Mana gekommen.

Magische Edelsteine werden auf Spontanmagie zurückgeführt.

\section{Begriffe}

Im Laufe der Jahre haben sich Begriffe eingebürgert, die erfahrenen Spielern geläufig sind. Die Liste soll dir die Möglichkeit geben, diese zu verstehen. Die Liste ist natürlich nicht vollständig.

Alte Sprache

Deutsch

Die Fünf

Die fünf alten Götter.

Gemmen, das

Eine Waffe mit magischen Steinen (engl gem) aufwerten

Gobaith

Über 12 Jahre (bis zum 21.12.2012) spielet Illarion auf dieser Insel.

Grüße

Hallo, Guten Tag, Gängigste Grußformel in Illarion

Kiste

Depot, die gelben Kisten für Items

Kreuz

Respawnpunkt

Zum Kreuz schicken = Töten

Neue Sprache

Englisch

Taube schicken

Nachricht schicken über PM

Wolken, das

Getötet werden [11]

Zwergentag

Real Life Tag (dito Woche, Monat, Jahr)

Crimson Order  Militärisch organisierter Ritterorden Ansässig in Cadomyr

Iron Watch Wächtermiliz der Stadt Galmair Ansässig in Galmair

Träger des Feuers  Religiöse Gruppierung, huldigt dem Gott des Feuers Bràgon

Ansässig in Runewick

[1] Notwendig ist ganz klar auch, wenn dir eine Bedienung nicht klar ist. Das akzeptiert jeder gute Spieler.  2 Zu nutzen, wenn du einen Fehler in Mantis melden willst. 3 Das System ist noch nicht vollständig implementiert. Derzeit kann jeder jede Sprache verstehen. Das muss nicht bedeuten, dass der Charakter das auch kann.

Wenn du Englisch nicht oder nicht gut kann, lass dich nicht verwirren. Häufig sprechen viele deutsche Spieler englisch sobald ein englischsprachiger Spieler in der Nähe ist. Grüße einfach auf Deutsch und gib zu verstehen, dass du der neuen Sprache nicht mächtig bist.  5

Rollenspieltechnisch schöner sind natürlich Ansprachen wie: \"Ich habe hier was zu ver

Willst du mit mir

[4] Drag und Drop mit gedrückter Shift Taste 8 Die übrigen sind so wertvoll, dass man sicher einen Spieler findet, der deinen Charakter auf Knien anfleht, es ihm zu verkaufen.

Max ein freies Feld dazwischen

Gerade für Neulinge bieten sich die folgenden Archetypen an:

Knecht oder Magd, gerade rausgeschmissen wegen irgendeines minderen Vergehens.

Handwerker oder Söldner ohne abgeschlossene Ausbildung auf Wanderschaft

[7] Z.B. Ein Charakter, der ein Schwert mit Level 70 führt, bisher aber nur Level 46 erreicht hat, wird dieses weder besonders effektiv führen noch besonders schnell lernen.

Ein Charakter der eine leichte Rüstung mit Level 10 trägt und bereits Level 12 erreicht hat wird beißende Hunde sehr effektiv abwehren und schnell neues lernen.  12

Angeblich sind auch schon magische Diamanten aufgetaucht.

[9] Spielintern als Zwergentag benannt.  16

[10] Es gibt auch innerhalb Illarions verschiedene konkurrierende Deutungen der Phänomene. In diesem Abschnitt sind die allgemein gültigen Grundlagen dargelegt.  19

Gemäß allgemeingültiger Lehrmeinung beinhalten lebende Wesen mehr Mana als leblose Dinge. Die Kapazität, die jedem Magier zur Verfügung steht um Mana zu beeinflussen wird landläufig auch als Mana bezeichnet. Es handelt sich natürlich nicht um dasselbe wie die Umgebungsmana. Zur Verwirrung der ungebildeten Bevölkerung ist diese Überschneidung vielen Magiern nur recht und billig.  21 Runenmagie wird in einigen Magieschulen als Unterart der Ritualmagie angesehen.  22 Im Rahmen des Untergangs von Gobaith haben die Götter die Stärke der magischen Reflexe nahezu verschwinden lassen. Nachdem eine Zeitlang Runenmagie außerhalb der Gegend rund um das Gasthaus zur Hanfschlinge noch möglich war weitete sich im Mas des Jahres 40 der Bereich auf die gesamte bekannte Welt aus.

In einem alten Client wurde das Bild des Chars eine Wolke, wenn er tot war.


\end{document}