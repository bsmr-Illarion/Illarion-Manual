\documentclass[a4paper,11pt]{scrreprt} 
%\usepackage[ngerman]{babel}
\usepackage[utf8x]{inputenc} 

%\usepackage{etoolbox}
%\usepackage[svgnames]{xcolor}

%\usepackage{tikz}

%\usepackage{framed}

\usepackage{libertine} % or any other font package
%\newcommand*\quotefont{\fontfamily{LinuxLibertineT-LF}} % selects Libertine as the quote font


\usepackage{fullpage}

\makeindex 
\begin{document}
\tableofcontents
\chapter{Getting started}
\section{Roleplaying}

There are numerous opinions and lessons describing role play. This one doesn't claim to be exhaustive nor sole validity. Always remember:
\begin{center}
\fbox{
\emph{You play a role. You are not the character you are playing.}}
\end{center}

The character you are playing should have the following characteristics: 
\begin{table}[h]
\begin{tabular}{ l p{12.5cm}}
Background story & You don't need a novel but some basic points should be 
defined. The empty parts in your story will fill in with time by 
itself. \\
(Dis-) Advantages & Nobody is only a bright hero and nobody loses all the 
time. Your character should have some positive as well as some negative attributes. \\
Goals & Your character should have a dream, something he or she 
wants to achieve desperately. \\
Phobia & There should be something he or she hates or fears. \\
Distinctiveness & Your character should have a quirk. It should be some distinctive feature, e.g. that could be a tiny voice, a 
special greeting or a preference for yellow hats. 
\end{tabular}
\end{table}

Please bear in mind that your character starts in Illarion unskilled. He or she cannot be an experienced crafter or warrior from the very beginning\footnote{Especially for players with less experience with roleplaying games, the following archetypes are useful: A servant or maid, recently thrown out of service due to any minor delinquency or a crafter or mercenary in apprenticeship on walk.}.
Give your counterpart a chance for reaction instead of forcing him or her into a situation. Don't pull her on her sleeve, \emph{try} to pull instead. A good role player will know how wo react properly.

Play in a way that everybody has fun with the game as well.

Don't force anybody to do something the player behind the character doesn't want to do. Especially thievery or robberies are difficult to play. It requires at least two persons having 
fun with. 

It is much more difficult to play a villain compared to a normal character. At least you need to 
agree that there finally will be a punishment he or she deserves. 

Don't let bluff yourself. Behind a grumpy dwarf, bloodthirsty orc or insensible mage is a friend 
you may steal horses with. 

New in game? \\
Ask game masters or player if there is something you don't know!
Chat with us in game, in our IRC chat or via our forum. 
\\

You think you don't know how to role play? \\
Everybody has been a beginner anytime. 
Nobody is perfect. 

You may find a larger role play advice as well as background stories on our web page. 

\section{Story/Setting/Lore}
\subsection{Factions}
Illarion's people are divided into three different factions. Every inhabitant of the world of Illarion has to belong to either of these---or be entirely factionless. Every faction is based in a distinct, name-giving town. Each of these towns have crafts being associated with them while other crafts are missing entirely in these towns.
\paragraph{Cadomyr}
Cadomyr is a desert stronghold in the southwest of Illarion and it was founded by humans and lizards. It is the town of honor and knights, earls, baroness and knights should feel 
well settled here. It is ruled by her majesty Queen Rosaline Edwards.

Cadomyr's colors are red and white, its emblem shows a horned lion. 

\paragraph{Galmair}
Galmair is a town in the mountain town in the north of Illarion. Its first settlers were dwaves and 
orcs. It's the town of money; smiths, merchants and miner prefer this settlement. It is ruled by Don Vallerio 
Guillamo.

Galmair's colors are black and yellow, its emblem shows a winged pig. 

\paragraph{Runewick}
Runewick is a sea town in the south east of Illarion and was founded by halflings and elves.
It is the town of knowledge, scholars and their students can find enough libraries around. 
It is ruled by Archmage Elvaine Morgan 

Runewick's colors are blue and silver and the emblem shows a double headed eagle.

\subsection{Races}
\paragraph{Humans}
\paragraph{Elves}
\paragraph{Dwarves}
\paragraph{Halflings}
\paragraph{Lizardmen}


\section{System requirements}
\section{Create an account}
Loginname: 5-34 characters
Meanings
\section{Create a character}
\subsection{Races, Attributes, Skills and all that}
\section{Install the client and log in}

\chapter{Controls}
\section{Moving}
You can move your character (walk or run) across the map by mouse or by keyboard. 
The character follows the mouse pointer whenever the left mouse button is pressed and held down. Mostly, he or she will 
avoid obstacles automatically. 
Double click  on a free tile (meadow, way, rock, plaster) and your character will start walking walking
to that tile. 
To exactly move your character try using the keyboard:
\begin{table}[h]
\begin{tabular}{ l p{14.5cm}}
W, A, S, D & Main directions \\
Arrow keys & Main directions. Intermediate directions are possible using 2 keys together  \\
Numpad & Main and intermediate directions \\
ALT and arrow key & Character turns without walking \\
ALT GR and arrow key & Character runs 
\end{tabular}
\end{table}

Beware: Your character can't run if he or she is overloaded (a heavy load slows you 
down). 
Some keyboard combinations could be used by your system alternatively 
(e.g. Alt Gr and up arrow turns the screen upside down)

\section{Talking/Chat}
Illarion emphasizes real roleplay. The basic interaction with other players as well as with NPCs 
takes place in the chat mode. The RETURN key switches the chat mode on and off.

There are several different chat modes which can be enabled by typing \# and some command in front of your text:
\begin{table}[h]
\begin{tabular}{ l p{14.5cm}}
\#me & Character descriptions. 
You show unaudible emotions or descriptions of your character. The text is displayed in yellow. 
e.g.  \#me laughs; \#me stinks as if he came out of a cesspit \\
\#w & Whispering. Your words can be heard to the next but one tile. Whispered text is shown in Grey. \\
\#s & Shout. 
Your voice can be heard over a long distance. The text will be displayed in red. \\
\#o & OOC (Out of Character). 
You use this to indicate that it is not your character talking but the player behind it. The text 
is displayed in double brackets in grey. Only characters standing close by can hear this kind of comments.
\end{tabular}
\end{table}

\section{Use}
\section{Fight}
\section{Trade}

\chapter{Game mechanics}
\section{Gathering}
\section{Crafting}
\section{Druidry}
\section{Trasures, Gems, Explorers Guild}

\chapter{(FAQ)}
\end{document}